\documentclass[12pt, a4paper]{article}
\usepackage{amsthm,amsfonts,amsmath,amssymb,amscd}
\usepackage[T2A]{fontenc}                         
\usepackage[utf8]{inputenc}                      
\usepackage[english, russian]{babel}
\usepackage{graphicx}
\usepackage{indentfirst}
\usepackage{multicol}
%\usepackage{cite} 
%\usepackage{psfrag}
%\IfFileExists{pscyr.sty}{\usepackage{pscyr}}{}    %

\usepackage[top=2cm,bottom=2cm,left=2cm,right=2cm]{geometry}

\linespread{1.5} 
\begin{document}
\begin{center}
Отчет о проделанной работе от 30.09.2022

\textbf{<<Разработка комбинированной математической модели распространения COVID-19 с учетом экономических агентов>>}

Воробьева Ирина
\end{center}


В предыдущем отчете от 30.08.2022 была описана идея построения комбинированной модели распространения COVID-19 с учетом экономических агентов. В рамках модели предлагается рассмотреть модель межотраслевого баланса для отдельно взятого региона и, с помощью шоков, наблюдать изменение таких макроэкономических характеристик, как Валовая добавленная стоимость (ВДС) в регионе. В качестве шоков предлагается рассматривать последствия пандемии COVID-19 и мер по борьбе с распространением пандемии: сметрность среди работников, локдауны.

Для построение модели межотраслевого баланса необходимо иметь симметричную матрицу <<Затраты-Выпуск>>, описывающую денежные потоки между отраслями. Такая матрица позволяет учитывать взаимосвязь отраслей в экономике и учитывать распространение шоков при воздействии на отдельные отрасли. Матрицы такого вида собираются Росстатом раз в 5 лет, причем матрицы строятся для экономики государства в целом, а межотраслевой баланс для отдельных регионов не строится.

Таким образом, для продолжения работы над моделью необходимо найти алгоритмы, позволяющие выделять связи между отраслями и строить соответствующие матрицы для отдельных регионов.

\section{Построение моделей межотраслевого баланса на региональном уровне}

Вопросы разработки моделей межотраслевого баланса (МОБ) на региональном уровне активно разрабатывается в различных странах мира. Среди подходов к решению данной задачи можно выделить два наиболее популярных:
\begin{itemize}
\item построение матрицы межотраслевого баланса на основе статистических данных;
\item построение матрицы межотраслевого баланса с помощью мультипликаторов, приводящих матрицы Росстата к региональным матрицам.
\end{itemize}

Первый подход опирается на открытые статистические данные Росстата по регионам и на закрытые данные в форме <<1-предприятие>>. Результаты применения такого подхода описаны, например, описан в статьях \cite{MOB_stat1} и \cite{MOB_stat2}. В форме <<1-предприятие>> средние и большие компании предоставляют данные о структуре своих расходов. Данные агрегируются по отраслям и формируют матрицу межотраслевого взаимодействия. Такой метод имеет следующие недостатки:
\begin{itemize}
\item в соответствии с российским законодательством органы государственной статистики не вправе публиковать информацию о работе той или иной отрасли в случае, если отчетность подают менее трех организаций (такие случаи встречаются, причем разные в разных регионах);
\item в форме <<1-предприятие>> присутствуют недостаточно детализированные статьи расходов, что вынуждает прибегать к федеральной модели;
\item для применения модели ко множеству регионов необходимо вручную обрабатывать большие объемы информации, запрашивать доступ к платным формам <<1-предприятие>>, учитывать п.1 из приведенных ранее для каждого отдельного региона.
\end{itemize}

Второй подход предполагает использование коэффициентов локализации (Location quotient - LQ), позволяющих приводить федеральную матрицу межотраслевого баланса к региональным. Коэффициенты локализации формируются из отношения занятости в отрасли на региональном и национальном уровне. Таким образом, коэффициенты позволяют учитывать долю федеральных производственных средств, сосредоточенных в рассматриваемом регионе. Такая модель позволяет при наличии данных о занятости по отраслям в регионе и в государстве быстро и эффективно строить балансы для отдельных субъектов федерации. Главным ограничением в данной модели является предположение о том, что структура производств одной отрасли в различных регионах одинакова. Однако, существует множество различных модификаций модели, позволяющих отходить от данного предположения.

Таким образом, хотя первый подход опирается на реальные статистические данные, имеющейся детализации недостаточно для того, чтобы полностью отказаться от привязки к федеральной модели, в то время как обработка существующей статистики достаточно затратна по времени и требует больших усилий при желании рассмотреть новый регион или выбрать другой год. Вторая модель, хотя и привязана к федеральной, и связана предположением о схожей структуре производственных процессов в различных регионах, позволяет быстро и достаточно точно построить матрицы межотраслевых балансов для различных регионов по небольшим объемам статистических данных.

\section{Построение матриц межотраслевых балансов с помощью коэффициентов локализации}

В данной модели для нормировки федеральных балансов используются коэффициенты локализации. Они могут строиться по различным статистическим данным о выпусках компаний, но наиболее популярными являются данные о занятости в различных отраслях, так как они считаются наиболее точными. Применение данного метода для различных субъектах Российской Федерации приведено в \cite{MOB_loc_base}. Также в рамках данного исследования опубликованы результаты расчетов с промежуточными коэффициентами и исходными данными по различным регионам: 
http://noo1.ranepa.ru/files/DB\_RMOB\_2017.xlsx. 

В рассматриваемом методе элементы регионального межотраслевого баланса $z_{ij}^R$ получаются из произведения регионального отраслевого выпуска $x_j^R$ и регионального технологического коэффициента $a_{ij}^R$. Технологический коэффициент можно интерпретировать как количество единиц региональной продукции $i$ для производства единицы региональной продукции $j$. Из предположения о схожей структуре производства в рамках одного продукта в различных регионах, можно получить формулу для определения $x_{j}^R$ из национального отраслевого выпуска:
$$
x_j^R = \dfrac{L_j^R}{L_j^N}x_j^N,
$$
где $L_j^R,\ L_j^N$ --- региональная и национальная занятость в отрасли $j$.

Коэффициент локализации $a_{ij}^R$ строится путем нормировки национальных коэффициентов: 
$$a_{ij}^R = \left\{ \begin{aligned}&t_{ij}a_{ij}^N,\quad t_{ij} \leqslant 1,\\&a_{ij}^N,\quad t_{ij} > 1\end{aligned}\right.$$. 

Коэффициент $t_{ij}$ отвечает за локализацию. Существуют различные подходы к его построению, далее будем рассматривать тот, который дает наиболее точные результаты \cite{MOB_Accur}, \cite{MOB_Monte_Carlo}.

Введем следующие обозначения: $L^R$ --- общая занятость в регионе, $L^N$ --- общая занятость на национальном уровне, $L_i^R,\ L_i^N$ --- общая занятость в $i$-й отрасли на региональном и национальном уровне соответственно, $\beta \geqslant 1$ --- эндогенный параметр. При таких обозначениях искомый коэффициент $t_{ij}$ равен
$$
t_{ij} = \mathrm{FLQ}_{ij} = \left\{\begin{aligned}
& \dfrac{L_i^R L_j^N}{L_i^N L_j^R}\lambda_r^\beta,\quad i \neq j,\\
& \dfrac{L_i^R l^N}{L^R L_i^N}\lambda_r^\beta,\quad i=j,
\end{aligned}\right.
$$
где $\lambda = \dfrac{L^R}{L^N}\dfrac{1}{\log_2\left[1 + \dfrac{L^R}{L^N}\right]}$.

Таким образом, для построения региональных матриц МОБ необходимо знать занятость по отраслям внутри регионов за интересующий год, федеральные матрицы МОБ и выпуски по отраслям. Все эти данные предоставлены Росстатом, что позволяет строить региональные матрицы с учетом актуальной информации о занятости населения. Важным преимуществом такого метода является возможность моделировать шоки в системе, связанные с работоспособностью населения, появляющиеся вследствие пандемии COVID-19: снижение количества занятых оказывает влияние на структуру межотраслевого взаимодействия и, как следствие, на экономику региона в целом.
\newpage
\begin{thebibliography}{00}
\bibitem{MOB_stat1}
Коган~А.~Б., <<Межотраслевой анализ экономики Новосибирской области>>, Вестник НГУЭУ, 2015.
\bibitem{MOB_stat2}
Котова~Т.~Е. <<Оценка внешнеторговых эффектов в экономике Хабаровского
края на основе использования таблиц «затраты–выпуск» >> Пространственная
экономика. 2012. No 1. С. 43–68.
\bibitem{MOB_Accur}
Kronenberg~T. <<Regional input-output models and the treatment of imports in the European System of Accounts (ESA)>> Jahrbuch für Regionalwissenschaft. 2012. Vol. 32. Рр. 175-191.
\bibitem{MOB_Monte_Carlo}
Bonfiglio~A. <<On the Parameterization of Techniques for Representing Regional Economic Structures>> Economic System Research. 2009. Vol. 212. Pp. 115-127.
\bibitem{MOB_loc_base}
Пономарев~Ю.~Ю., Евдокимов~Д.~Ю., <<Построение усеченных таблиц <<Затраты-Выпуск>> для регионов России с использованием коэффициентов локализации>>, Проблемы прогнозирования, 2021, No 6.
\bibitem{COVID_model}
 Криворотько О.И., Кабанихин С.И., Сосновская М.И., Андорная Д.В. Анализ чувствительности и идентифицируемости математических моделей распространения эпидемии COVID-19. Вавиловский журнал генетики и селекции, \textbf{25}(1), 82--91 (2021).
\bibitem{COVID_code}
COVID-19 Agent-based Simulator \underline{https://github.com/InstituteforDiseaseModeling/covasim}
\bibitem{Region_stat}
Регионы России. Социально-экономические показатели - 2021 год \underline{https://gks.ru/bgd/regl/b21\_14p/Main.htm}
\bibitem{Rosstat_stat}
Росстат, таблицы <<затраты-выпуск>> --- 2016 год \underline{https://rosstat.gov.ru/statistics/accounts}
\end{thebibliography}
\end{document} 